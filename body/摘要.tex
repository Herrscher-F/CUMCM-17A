\begin{abstract}
本文针对CT系统的参数标定与图像重建问题,建立了一套完整的数学模型与算法。首先,基于Radon变换理论,利用已知模板的几何特性与投影数据,通过投影质心轨迹(TPC)法精确标定了系统的旋转中心、探测器间距和扫描角度。其次,应用滤波反投影(FBP)算法,对两个未知介质的投影数据进行高质量的图像重建。最后,对标定方案的精度与稳定性进行分析,并设计了一种性能更优的新型标定模板。

\textbf{针对问题一},本文旨在精确标定CT系统的三个关键几何参数,本文采用分步求解策略:
首先利用模板中圆形物体投影宽度恒定的特性,标定出探测器单元间距为\textbf{0.2774 mm};
结合投影宽度匹配法确定了全部180个扫描角度(见\textbf{表一})
最后结合投影数据的\textbf{质心轨迹},并将其与正弦模型进行最小二乘拟合,精确求解出系统旋转中心坐标为\textbf{(58.5, 40.0) mm},

\textbf{针对问题二和问题三},本文的目标是利用已标定的参数重建未知介质的内部结构,采用了经典的\textbf{滤波反投影(FBP)算法}。
该方法以\textbf{傅里叶中心切片定理}为基础,通过对每个角度的投影数据应用一个“斜坡滤波器”与“汉明窗”的组合滤波器进行预处理,
有效增强了图像细节并抑制了噪声伪影,最终成功重建了两个未知介质清晰的\textbf{二维吸收率分布图像},并提取了指定位置的吸收率值。

\textbf{针对问题四},本文分析了标定模型的精度与稳定性,并提出优化方案。
我们设计了基于结构相似性(SSIM)和参数扰动的灵敏度分析框架来评估原模型。
基于分析提出了一种由四个非对称分布的高密度微珠组成的新型标定模板,该设计通过引入更强的\textbf{几何约束和尖锐特征},
能够从根本上提升标定过程的精度和鲁棒性,并建立了一个\textbf{全局约束优化模型}。

\keywords{滤波反投影(FBP)、投影质心轨迹(TPC)、Radon变换}
\end{abstract}


