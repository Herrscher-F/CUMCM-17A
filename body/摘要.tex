\begin{abstract}
\begin{quote}    
首页三要素: 论文标题 + 摘要 + 关键词
\begin{enumerate}
    \item 标题:
    \begin{itemize}
        \item 基于所使用的主要模型或者方法作为标题(推荐)
        \item 直接使用赛题所给的题目或者要研究的问题作为标题
    \end{itemize}
    \item 摘要:
    摘要是数模论文写作中最重要的一部分,因为评阅老师的时间有限,拿到一篇论文后不会完整的从头读到尾,所以评阅老师往往会重点阅读摘要部分,并结合官方的评阅要点来对你的论文进行初步评定。因此,大家一定要好好打磨论文的摘要,摘要一般是其他部分都完成后再来书写,写完后需要反复阅读反复修改。
    \item 关键词:
    关键词一般放4-6个,可以放论文中使用的主要模型,也可以放论文里面出现次数较多,能体现论文的主要内容的词。
\end{enumerate}
\end{quote}
开头段:需要充分概括论文内容,一般两到三句话即可,长度控制在三至五行。

\textbf{针对问题一}\cite{ref1},解决了什么问题;应用了什么方法;得到了什么结果。\par

\textbf{针对问题二}\cite{ref2},同上。\par

\textbf{针对问题三}\cite{ref3},同上。\par

\keywords{关键词1、关键词2、关键词3}
\end{abstract}
