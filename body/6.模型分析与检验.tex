\section{模型分析与检验}
模型的分析与检验的内容也可以放到模型的建立与求解部分,这里我们单独抽出来进行讲解,因为这部分往往是论文的加分项,很多优秀论文也会单独抽出一节来对这个内容进行讨论。\par
模型的分析 :在建模比赛中模型分析主要有两种,一个是灵敏度(性)分析,另一个是误差分析。灵敏度分析是研究与分析一个系统(或模型)的状态或输出变化对系统参数或周围条件变化的敏感程度的方法。其通用的步骤是:控制其他参数不变的情况下,改变模型中某个重要参数的值,然后观察模型的结果的变化情况。误差分析是指分析模型中的误差来源,或者估算模型中存在的误差,一般用于预测问题或者数值计算类问题。\par
模型的检验:模型检验可以分为两种,一种是使用模型之前应该进行的检验,例如层次分析法中一致性检验,灰色预测中的准指数规律的检验,这部分内容应该放在模型的建立部分;另一种是使用了模型后对模型的结果进行检验,数模中最常见的是稳定性检验,实际上这里的稳定性检验和前面的灵敏度分析非常类似。









