\subsection{问题一}
\subsubsection{探测器间距求解}

探测器单元间距$\Delta\rho$的精确标定是后续图像重建的基础。
本文利用标定模板中已知的精确几何尺寸与附件2中实测投影数据(正弦图)之间的对应关系来求解该参数。
核心思想在于,通过识别特定投影角度下的模板投影宽度,
建立物理尺寸(单位:mm)与探测器阵列测量宽度之间的比例关系。\par


由椭圆的几何特性和圆的对称性可知,椭圆的投影宽度随扫描角度变化,而圆的投影宽度恒定。
因此,我们可以利用圆的投影宽度恒定这一特性来求解探测器间距。\par


最终求解:
将圆形的恒定直径 $D_circle$ 和计算得到的平均探测器单元数 $N_avg$ 代入初始公式,最终确定探测器单元间距 $D_circle$:

$$\Delta\rho = \frac{D_{circle}}{N_{avg}} = \frac{4}{N_{avg}}$$

通过遍历所有180个角度的投影数据,我们可以找到对应的最大投影宽度和最小投影宽度,
具体过程如下:\par
$step1.$ 按照角度划分数据,共180组;\par
$step2.$ 按照0与非0对每组的数据进行阈值划分;\par
$step3.$ 利用梯度信息寻找物体边界;\par
$step4.$ 根据边界信息计算投影长度;\par
$step5.$ 寻找始终不变的投影长度。(圆的投影长度恒定)\par
$step6.$ 计算平均值,得到探测器间距。


\subsubsection{旋转中心求解}

一个刚性物体在围绕一个固定中心旋转时,其投影到一维探测器上的质量中心的运动轨迹是一条正弦曲线。这条正弦曲线的振幅、相位和直流偏置,唯一地决定了物体的质心相对于旋转中心的位置。利用质心轨迹来求解旋转中心的方法被称为“投影质心轨迹法”\par
在本题中,我们将投影的“质量中心”替换为“吸收率中心”,并利用吸收率中心的运动轨迹来求解旋转中心。\par
具体过程如下:
\begin{enumerate}
    \item \textbf{计算模板的吸收率中心:} 首先,在托盘坐标系(以托盘左下角为原点)中,计算标定模板的吸收率中心 $(x_m,y_m)$。
    \item \textbf{计算投影吸收率中心轨迹(TPC):} 对附件2中的180组投影数据,计算每一组投影的吸收率中心。对于第 $j$ 组投影(对应角度 $\theta_j$),其投影数据为 $p_j(i)$,$i=1,2,...,512$。其投影吸收率中心的像素位置 $C_j$ 计算如下:
       $$C_j=\frac{\sum_{i=1}^{512}i\cdot p_j(i)}{\sum_{i=1}^{512}p_j(i)}$$
       计算所有180个角度的投影吸收率中心,得到一个包含180个点的时间序列 $[C_1,C_2,...,C_{180}]$,这就是投影吸收率中心轨迹(TPC)。
    \item \textbf{正弦拟合:} 理论上,TPC应满足以下正弦模型:
       $$C(\theta)=A\sin(\theta+\phi)+C_{offset}$$
       使用非线性最小二乘法将计算得到的TPC数据点 $\{(\theta_j,C_j)\}_{j=1}^{180}$ 拟合到上述模型,求解出最佳的振幅 $A$、相位 $\phi$ 和直流偏置 $C_{offset}$。在拟合前,需要对角度 $\theta_j$ 做出初步假设,例如,假设为等间隔的 $0°,1°,...,179°$。这个初步假设的误差主要影响相位 $\phi$,后续可以被校正。
    \item \textbf{求解旋转中心 $(x_c,y_c)$:} 拟合得到的参数与旋转中心坐标 $(x_c,y_c)$ 之间存在直接的几何关系。设物体的吸收率中心为 $(x_m,y_m)$,旋转中心为 $(x_c,y_c)$。物体吸收率中心相对于旋转中心的坐标为 $(\Delta x,\Delta y)=(x_m-x_c,y_m-y_c)$。其极坐标表示为 $(R,\alpha)$,其中 $R=\sqrt{\Delta x^2+\Delta y^2}$,$\alpha=\text{atan2}(\Delta y,\Delta x)$。
       
       投影吸收率中心的物理位置 $C_{phys}(\theta)$ 与这些参数的关系是:
       $$C_{phys}(\theta)=R\sin(\theta-\alpha)+\rho_c(\theta)$$
       其中 $\rho_c(\theta)$ 是旋转中心在投影方向上的坐标。对于平行束,$\rho_c(\theta)$ 是一个常数,即旋转中心到探测器中心线的投影偏移。
       
       将像素位置 $C_j$ 转换为物理位置 $C_{phys,j}=(C_j-i_{center})\cdot\Delta d$,其中 $i_{center}$ 是探测器中心像素的索引(通常为256.5)。拟合后的模型变为:
       $$C_{phys}(\theta)=(A\cdot\Delta d)\sin(\theta+\phi)+(C_{offset}-i_{center})\cdot\Delta d$$
       通过对比,可以建立如下关系:
       \begin{itemize}
           \item 振幅:$A\cdot\Delta d=R=\sqrt{(x_m-x_c)^2+(y_m-y_c)^2}$
           \item 相位:$\phi=-\alpha=-\text{atan2}(y_m-y_c,x_m-x_c)$
           \item 偏置:$(C_{offset}-i_{center})\cdot\Delta d$ 代表了旋转中心在探测器坐标系中的平均偏移。
       \end{itemize}
\end{enumerate}

通过解这个方程组,并结合已知的模板吸收率中心 $(x_m,y_m)$,即可反解出旋转中心 $(x_c,y_c)$。

\subsubsection{投影角度集求解}

在平行束CT中,完整的投影数据需要在 $180°$ 范围内采集,因此可以合理假设角度增量 $\Delta\theta=1.0°$。
标定的关键在于确定初始角度 $\theta_0$。
而我们在求解旋转中心时得到的相位 $\phi$ 实际已经包含了初始角度的信息。
$\phi$ 直接关联了物体质心相对于旋转中心的初始方位角。
通过几何关系,可以从 $\phi$ 和 $(x_c,y_c)$ 推算出 $\theta_0$。  
