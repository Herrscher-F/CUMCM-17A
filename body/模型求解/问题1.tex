\subsection{问题一}
\subsubsection{探测器间距求解}

探测器单元间距$\Delta\rho$的精确标定是后续图像重建的基础。
本文利用标定模板中已知的精确几何尺寸与附件2中实测投影数据(正弦图)之间的对应关系来求解该参数。
核心思想在于,通过识别特定投影角度下的模板投影宽度,
建立物理尺寸(单位:mm)与探测器阵列测量宽度之间的比例关系。\par


由椭圆的几何特性和圆的对称性可知,椭圆的投影宽度随扫描角度变化,而圆的投影宽度恒定。
因此,我们可以利用圆的投影宽度恒定这一特性来求解探测器间距。\par


最终求解:
将圆形的恒定直径 $D_circle$ 和计算得到的平均探测器单元数 $N_avg$ 代入初始公式,最终确定探测器单元间距 $D_circle$:

$$\Delta\rho = \frac{D_{circle}}{N_{avg}} = \frac{4}{N_{avg}}$$

通过遍历所有180个角度的投影数据,我们可以找到对应的最大投影宽度和最小投影宽度,
具体过程如下:\par
$step1.$ 按照角度划分数据,共180组;\par
$step2.$ 按照0与非0对每组的数据进行阈值划分;\par
$step3.$ 利用梯度信息寻找物体边界;\par
$step4.$ 根据边界信息计算投影长度;\par
$step5.$ 寻找始终不变的投影长度。(圆的投影长度恒定)\par
$step6.$ 计算平均值,得到探测器间距。

通过对180组投影数据的分析,我们共识别出265个有效的物体投影片段,其中85个片段对应于模板中的圆形物体。基于圆形物体投影长度的高度一致性(平均长度为28.8353像素,第一四分位数为29像素),我们精确地标定出探测器单元的物理间距为0.2774毫米。


\subsubsection{投影角度集求解}

\begin{enumerate}
    \item 按180个未知旋转角度对附件2的数据进行划分:
    \begin{itemize}
        \item 对于每个角度,先按0与非0进行阈值划分
        \item 计算该角度下的投影长度或者说"第一个非零值探测器到最后一个非零值探测器之间的距离"(由于模板由两个物体组成,因此投影可能是多段的,对于这种投影,它的长度需要定义为一个向量,表示每段的投影长度)
        \item 得到180个投影数据,这些数据有可能是数值,也有可能是长度不定的向量,具体由投影的段数决定
        \item 将单位由探测器间距改为mm
    \end{itemize}

    \item 建立几何模型计算理论投影长度:
    \begin{itemize}
        \item 根据附件1的几何信息,建立平面直角坐标系(以mm为单位,像素大小为0.3922mm)
        \item 已知模板由标准椭圆和圆形组成,建立其数学表达式:
        \begin{itemize}
            \item 设椭圆中心为$(x_e, y_e)$,长轴$a$,短轴$b$,椭圆方程为:
            $$\frac{(x-x_e)^2}{a^2} + \frac{(y-y_e)^2}{b^2} = 1$$
            \item 设圆心为$(x_c, y_c)$,半径为$r$,圆的方程为:
            $$(x-x_c)^2 + (y-y_c)^2 = r^2$$
        \end{itemize}
        \item 对于3600个预设角度$\theta_i$,利用几何方法计算投影长度:
        \begin{itemize}
            \item 对于椭圆:计算其在角度$\theta_i$方向上的投影宽度
            $$L_{ellipse}(\theta) = 2\sqrt{a^2\cos^2(\theta) + b^2\sin^2(\theta)}$$
            \item 对于圆:其投影宽度恒定为$L_{circle} = 2r$
            \item 根据两个几何体的相对位置,确定投影是连续的还是分离的
        \end{itemize}
        \item 得到3600个投影长度数据(以mm为单位)
    \end{itemize}

    \item 对于每个实际投影长度数据,从3600个预设角度中选择与之最为匹配的:
    \begin{itemize}
        \item 投影长度维数必须一致
        \item 将向量的值按从小到大重新排序
        \item 选择向量距离最短的那个作为匹配的角度
    \end{itemize}
\end{enumerate}

编程进行计算,得到的180个角度数据如下:
\begin{table}[H]
    \centering
    \caption{180个投影角度数据(单位:度)}
    \label{tab:projection_angles} % 添加一个标签,方便在正文中引用
    \begin{tabular}{cccccccccc}
        \toprule
        150.3 & 148.7 & 148.7 & 147.1 & 146.3 & 145.5 & 143.9 & 143.1 & 142.3 & 140.7 \\
        139.9 & 139.1 & 138.3 & 137.0 & 127.1 & 137.0 & 135.2 & 135.2 & 133.4 & 131.6 \\
        130.7 & 129.8 & 131.6 & 137.0 & 126.2 & 125.3 & 124.4 & 122.6 & 121.7 & 120.8 \\
        119.9 & 119.0 & 118.1 & 117.2 & 116.3 & 114.5 & 114.5 & 112.7 & 112.7 & 110.9 \\
        110.0 & 109.1 & 108.2 & 106.4 & 105.5 & 104.6 & 103.7 & 101.9 & 101.0 & 101.0 \\
        99.2 & 82.1 & 83.9 & 83.9 & 86.6 & 86.6 & 86.6 & 90.2 & 90.2 & 90.2 \\
        90.2 & 90.2 & 90.2 & 90.2 & 90.2 & 86.6 & 86.6 & 83.9 & 83.9 & 83.9 \\
        82.1 & 80.3 & 79.4 & 77.6 & 76.7 & 75.8 & 74.9 & 74.0 & 72.2 & 72.2 \\
        71.3 & 70.4 & 68.6 & 67.7 & 66.8 & 65.9 & 64.1 & 64.1 & 62.3 & 61.4 \\
        60.5 & 59.6 & 58.7 & 57.8 & 56.9 & 56.0 & 55.1 & 53.3 & 43.4 & 49.7 \\
        49.7 & 49.7 & 47.0 & 47.0 & 45.2 & 45.2 & 43.4 & 53.3 & 42.4 & 41.6 \\
        40.0 & 40.0 & 38.4 & 37.6 & 36.8 & 35.2 & 34.4 & 33.6 & 32.0 & 32.0 \\
        30.4 & 29.6 & 28.8 & 27.2 & 26.4 & 25.6 & 24.8 & 23.2 & 22.4 & 20.8 \\
        20.8 & 19.2 & 18.4 & 17.6 & 16.8 & 15.2 & 15.2 & 13.6 & 12.8 & 12.0 \\
        10.4 & 9.6 & 8.8 & 8.0 & 6.4 & 5.6 & 4.0 & 4.0 & 3.2 & 0.0 \\
        3.2 & 0.0 & 3.2 & 3.2 & 4.0 & 5.6 & 6.4 & 7.2 & 8.0 & 8.8 \\
        9.6 & 10.4 & 12.0 & 12.8 & 13.6 & 15.2 & 16.0 & 16.8 & 17.6 & 19.2 \\
        20.0 & 20.8 & 21.6 & 23.2 & 24.0 & 24.8 & 25.6 & 27.2 & 28.0 & 28.8 \\
        \bottomrule
    \end{tabular}
\end{table}

\subsubsection{旋转中心求解}

一个刚性物体在围绕一个固定中心旋转时,其投影到一维探测器上的质量中心的运动轨迹是一条正弦曲线。这条正弦曲线的振幅、相位和直流偏置,唯一地决定了物体的质心相对于旋转中心的位置。利用质心轨迹来求解旋转中心的方法被称为“投影质心轨迹法”\par
具体过程如下:
\begin{enumerate}
    \item \textbf{计算模板的吸收率中心:} 首先,在托盘坐标系(以托盘左下角为原点)中,计算标定模板的吸收率中心(质心) $(x_m,y_m)$。
    \item \textbf{计算投影吸收率中心轨迹(TPC):} 对附件2中的180组投影数据,计算每一组投影的吸收率中心。对于第 $j$ 组投影(对应角度 $\theta_j$),其投影数据为 $p_j(i)$,$i=1,2,...,512$。其投影吸收率中心的像素位置 $C_j$ 计算如下:
       $$C_j=\frac{\sum_{i=1}^{512}i\cdot p_j(i)}{\sum_{i=1}^{512}p_j(i)}$$
       计算所有180个角度的投影吸收率中心,得到一个时间序列$[C_1,C_2,...,C_{180}]$,这就是投影吸收率中心轨迹(TPC)。
    \item \textbf{正弦拟合:} 理论上,TPC应满足以下正弦模型:
       $$C(\theta)=A\sin(\theta+\phi)+C_{offset}$$
       使用非线性最小二乘法将计算得到的TPC数据点 $\{(\theta_j,C_j)\}_{j=1}^{180}$ 拟合到上述模型,求解出最佳的振幅 $A$、相位 $\phi$ 和直流偏置 $C_{offset}$。
    \item \textbf{求解旋转中心 $(x_c,y_c)$:} 拟合得到的参数与旋转中心坐标 $(x_c,y_c)$ 之间存在直接的几何关系。设物体的吸收率中心为 $(x_m,y_m)$,旋转中心为 $(x_c,y_c)$。物体吸收率中心相对于旋转中心的坐标为 $(\Delta x,\Delta y)=(x_m-x_c,y_m-y_c)$。其极坐标表示为 $(R,\alpha)$,其中 $R=\sqrt{\Delta x^2+\Delta y^2}$,$\alpha=\text{atan2}(\Delta y,\Delta x)$。
       投影吸收率中心的物理位置 $C_{phys}(\theta)$ 与这些参数的关系是:
       $$C_{phys}(\theta)=R\sin(\theta-\alpha)+\rho_c(\theta)$$
       其中 $\rho_c(\theta)$ 是旋转中心在投影方向上的坐标。对于平行束,$\rho_c(\theta)$ 是一个常数,即旋转中心到探测器中心线的投影偏移。
       将像素位置 $C_j$ 转换为物理位置 $C_{phys,j}=(C_j-i_{center})\cdot\Delta d$,其中 $i_{center}$ 是探测器中心像素的索引(通常为256.5)。拟合后的模型变为:
       $$C_{phys}(\theta)=(A\cdot\Delta d)\sin(\theta+\phi)+(C_{offset}-i_{center})\cdot\Delta d$$
       通过对比,可以建立如下关系:
       \begin{itemize}
           \item 振幅:$A\cdot\Delta d=R=\sqrt{(x_m-x_c)^2+(y_m-y_c)^2}$
           \item 相位:$\phi=-\alpha=-\text{atan2}(y_m-y_c,x_m-x_c)$
           \item 偏置:$(C_{offset}-i_{center})\cdot\Delta d$ 代表了旋转中心在探测器坐标系中的平均偏移。
       \end{itemize}
\end{enumerate}

通过解这个方程组,并结合已知的模板吸收率中心 $(x_m,y_m)$,即可反解出旋转中心 $(x_c,y_c)$。\par

以托盘左下角为原点建立平面直角坐标系(单位:mm),根据代码计算结果:\par
模板的吸收率中心为$(x_m,y_m)=(51.2,50)$,$A=45.34$,$\phi=-127.34$,$C_{offset}=261.13$。\par
根据几何关系,可以得到旋转中心为$(x_c,y_c)=(58.5,40.0)$。








