\subsection{问题四}
\subsubsection{灵敏度检验}
为量化评估模型的稳定性和精度,可以采取以下步骤:
\begin{itemize}
    \item \textbf{精度验证:} 使用第一部分标定的参数,对已知的标定模板(附件1)自身进行FBP重建。将重建图像与理论上的模板图像进行比较,计算均方误差(MSE)或结构相似性指数(SSIM)等指标,量化重建的保真度。(哪个指标表现好就用哪个指标)  
    \item \textbf{稳定性测试:} 在标定出的参数上人为引入微小的随机扰动(例如,给 $x_c$ 增加0.1 mm的误差),然后用受扰动的参数重新进行重建。观察重建图像质量的退化程度,特别是伪影的出现情况。通过多次实验,可以评估模型对各类参数误差的敏感度,即稳定性。
\end{itemize}


\subsubsection{设计方案}  
基于建模过程中的设计原则,我们提出一种新型标定模板。该模板由一个低吸收率的基座(如有机玻璃圆盘)和嵌入其中的四个高密度、小尺寸的钨钢微珠构成。这四个微珠的布局是非对称的,具体来说:
\begin{itemize}
    \item 三个微珠构成一个不等边直角三角形,顶点分别位于 $(x_1,y_1)$, $(x_2,y_2)$, $(x_3,y_3)$。  
    \item 第四个微珠放置在离中心较远的位置 $(x_4,y_4)$,以打破任何潜在的对称性,并为旋转半径的确定提供长力臂。
\end{itemize}
使用这种新模板,标定算法将变得更为强大和稳定:
\begin{itemize}
    \item \textbf{特征识别:} 在正弦图中,四个高对比度的微珠会形成四条清晰、独立的、高信噪比的正弦轨迹。  
    \item \textbf{独立拟合与联合优化:}  
    \begin{itemize}
        \item 对每一条轨迹 $k$($k=1,2,3,4$)单独进行正弦拟合,得到其振幅 $A_k$、相位 $\phi_k$ 和偏置 $C_{offset,k}$。  
        \item 建立一个联合优化模型。该模型的变量是全局的系统参数(旋转中心 $(x_c,y_c)$,初始角度 $\theta_0$ 等)。目标函数是最小化所有四条轨迹的理论模型与实际数据之间的总误差。  
        \item 该模型还包含强几何约束:任意两个微珠 $k_1,k_2$ 在重建空间中的距离必须等于其已知的物理距离 $D_{12}=\sqrt{(x_{k_1}-x_{k_2})^2+(y_{k_1}-y_{k_2})^2}$。这些约束关系可以通过拟合出的振幅和相位参数来表达。  
    \end{itemize}
\end{itemize}