\subsection{问题四}
\subsubsection{灵敏度检验}
为了定量地评估我们标定模型的精度与稳定性,我们设计了如下的灵敏度检验方案。
该方案不仅验证了标定结果的准确性,还探究了模型对参数扰动的鲁棒性。

\begin{enumerate}[label=(\arabic*), left=0.5em]
    \item \textbf{精度验证} \par
    利用问题一中标定出的系统参数,对已知的标定模板(附件1)自身执行一次完整的滤波反投影(FBP)重建。
    通过将重建得到的吸收率图像与理论上的原始模板图像进行像素级比较,我们可以计算两者的结构相似性指数(SSIM)。
    相比于均方误差(MSE),SSIM更能从亮度、对比度和结构三个方面综合评估图像的保真度,
    更符合人类视觉感知,因此是衡量重建质量的更优指标。一个接近于1的SSIM值,将证明我们的标定参数具有很高的准确性。
    \item \textbf{稳定性测试} \par
    以标定出的最优参数为基准,对每个参数(如旋转中心坐标 $x_c$, $y_c$)分别施加一系列可控的、微小的随机扰动。
    然后,我们使用这些受扰动的参数组对模板进行重建,
    并观察重建图像质量的退化情况,特别是特征性伪影(如环状伪影)的出现与强度。
    通过系统性地分析图像质量指标随参数扰动幅度的变化关系,我们可以绘制出模型的“敏感度曲线”,
    从而量化模型对各类参数误差的鲁棒性,即其稳定性。
\end{enumerate}

\subsubsection{设计方案}  
基于前述的设计原则,我们提出一种新型的高精度标定模板。
该模板由一个低吸收率的均匀基座(例如有机玻璃圆盘)和精确嵌入其中的四个高密度、小尺寸的钨钢微珠构成。
这些微珠的布局经过精心设计,以打破对称性并提供强大的几何约束,其具体坐标布局如下:\par
\begin{itemize}
    \item 三个微珠构成一个非对称的不等边直角三角形,其顶点坐标精确已知。
    \item 第四个微珠放置在离中心较远的位置,以提供一个长几何杠杆臂,这对于精确约束旋转中心和图像尺度至关重要。
\end{itemize}


采用这种新模板,标定算法的性能和鲁棒性将得到显著提升。其工作流程将转变为对投影数据中清晰特征的直接分析:
\begin{enumerate}[label=(\arabic*), left=0.5em]
    \item \textbf{特征识别与提取} \par
    在扫描得到的正弦图中,四个高对比度的微珠将形成四条清晰、独立且信噪比极高的正弦轨迹。
    我们可以通过简单的图像处理算法精确地提取出这四条轨迹的中心线数据。
    \item \textbf{独立拟合与联合优化} \par
    对每条正弦轨迹 $k$ ($k=1,2,3,4$) 单独进行正弦函数拟合,
    从而初步估计出其振幅 $A_k$、相位 $\phi_k$ 和直流偏置 $C_k$。
    这些参数直接对应于每个微珠相对于旋转中心的极坐标位置。
    \item \textbf{全局约束优化} \par
    建立一个全局联合优化模型。
    该模型的优化变量是全局系统参数,如旋转中心 $(x_c, y_c)$ 和初始扫描角 $\theta_0$。
    目标函数旨在最小化由这些全局参数计算出的四条理论正弦轨迹与实际提取数据之间的总误差。
    更重要的是,该模型引入了刚体几何约束:
    任意两个微珠 $k_i, k_j$ 在重建空间中的距离,必须严格等于它们已知的物理距离 $D_{ij}$。
    这些强约束条件极大地减少了解空间的自由度,确保了全局最优解的唯一性和准确性,
    从而实现了对系统参数前所未有的高精度标定。
\end{enumerate}