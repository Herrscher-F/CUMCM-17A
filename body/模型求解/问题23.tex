\subsection{问题二和问题三}
在问题(1)中完成对CT系统关键参数——旋转中心 $(x_c, y_c)$、探测器间距 $d$ 
以及180个扫描角度 $\{\theta_j\}$——的精确标定后,本文利用这些参数来重建未知介质的断层图像,
重建过程采用前述的滤波反投影(FBP)算法。

\subsubsection{投影数据坐标标定}
首先,我们将原始的投影数据(附件3)映射到标准的物理坐标系中,
形成规范化的正弦图 $p(\rho, \theta)$。该过程为投影数据的每一个数据点赋予了明确的物理意义:\par
角度坐标 $\theta_j$:\par
根据标定出的初始角度和角度增量,确定第 $j$ 个投影($j=1, \dots, 180$)对应的扫描角度。\par
空间坐标 $\rho_i$:\par
对于第 $i$ 个探测器单元($i=1, \dots, 512$),
其对应的投影射线到旋转中心的垂直距离为 $\rho_i = (i - i_c) \cdot d$。
其中,$i_c$ 是旋转中心在探测器阵列上的投影索引,可由标定参数计算得出。\par

\subsubsection{投影滤波}
此步骤旨在增强投影数据中的高频信息以补偿反投影过程中的模糊效应。
我们对正弦图中每一个角度 $\theta_j$ 下的投影数据 $p(\rho, \theta_j)$(即正弦图的每一行)执行以下操作:\par
傅里叶变换:\par
通过快速傅里叶变换(FFT)将其转换至频域,得到 $P(\omega, \theta_j)$。\par
频域滤波:\par
将 $P(\omega, \theta_j)$ 与我们设计的加汉明窗的斜坡滤波器 $H_{win}(\omega)$ 相乘,
得到滤波后的频域表示 $P_{filtered}(\omega, \theta_j) = P(\omega, \theta_j) \cdot H_{win}(\omega)$。\par
傅里叶逆变换:\par
通过快速傅里叶逆变换(IFFT)将 $P_{filtered}(\omega, \theta_j)$ 转换回空间域,
得到滤波后的投影数据 $p_{filtered}(\rho, \theta_j)$。\par

\subsubsection{反投影}
反投影是将滤波后的投影数据“涂抹”回图像空间,以重建吸收率分布的过程。\par
初始化图像:\par
我们创建一个尺寸为 `256 × 256` 的空白图像矩阵 $f(x, y)$,并将其所有像素值初始化为零。
该矩阵的坐标系与正方形托盘的物理坐标系一一对应。\par
像素值累加:\par
我们遍历图像矩阵中的每一个像素 $(x, y)$。
对于每个像素,我们再遍历所有180个扫描角度 $\theta_j$。
在每个角度下,我们计算该像素中心相对于CT旋转中心 $(x_c, y_c)$ 的投影坐标 $\rho'$:
$$
    \rho' = (x - x_c)\cos\theta_j + (y - y_c)\sin\theta_j
$$
由于计算出的 $\rho'$ 通常不正好落在探测器的采样点上,我们采用**线性插值法**,在滤波后的投影数据 $p_{filtered}(\rho, \theta_j)$ 中估计该位置的投影值。然后,将此插值结果累加到像素 $f(x, y)$ 上。为了使离散求和更好地逼近连续积分,累加时需乘以角度步长的弧度值 $\Delta\theta_{\text{rad}}$(即 $\pi/180$)。
$$
    f(x,y) \leftarrow f(x,y) + \text{Interpolate}(p_{filtered}(\cdot, \theta_j), \rho') \cdot \Delta\theta_{\text{rad}}
$$

\subsubsection{图像后处理与信息提取}
遍历完成后,矩阵 $f(x, y)$ 即为重建出的物体吸收率断层图像。
为了提取未知介质的几何信息,我们首先对重建图像进行可视化。
通过设定一个合适的阈值,可以将吸收率较高的物体区域与背景分离,
生成二值化图像,从而清晰地观察其轮廓和形状。
为确定其在托盘中的位置,我们计算二值化图像的几何质心,并将其作为未知介质的位置坐标。