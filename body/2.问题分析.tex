\section{问题分析}
\subsection{问题一}
问题一的核心任务是利用已知的标定模板(图2)及其投影数据(附件2),
对CT系统的三个关键几何参数——旋转中心坐标、探测器单元间距和180个扫描角度 —— 进行精确标定。
解决此问题的基本思路是,建立未知参数与投影数据可观测特征之间的数学关系,
通过对比理论模型与实际数据来反向求解参数。\par
本文计划采用分步求解的策略:
首先,利用模板中小圆其在任何角度下的投影宽度均为直径的几何特性,
通过在投影数据中识别并测量其所占据的探测器单元数量,即可标定出探测器单元的物理间距。
然后,我们需要应用投影质心轨迹法。
该方法指出,物体投影的质心位置会随着扫描角度的变化呈现标准的正弦曲线轨迹。
本文通过采用吸收率中心来
计算出附件2中每一行投影数据的质心位置,
然后利用最小二乘法将这180个质心点拟合成一条正弦曲线。
该正弦曲线的振幅、相位和直流偏置三个参数,与模板的质心、系统旋转中心以及初始扫描角度直接关联,
通过求解这一方程组,便可同时确定旋转中心的精确坐标和扫描的初始角度,进而推算出全部180个扫描方向。

\subsection{问题二和问题三}
本研究的核心任务分为两个层面:首先是利用已知几何特征的标定模板(附件一、二)对CT成像系统进行参数标定(问题一);其次是运用标定好的系统参数,对两个未知介质(附件三、五)进行高质量的图像重建(问题二、三),并对标定方案的精度和稳定性进行评估与优化(问题四)。其中,图像重建是整个任务的落脚点,其质量直接依赖于参数标定的准确性。

问题二与问题三的本质是一个典型的CT图像重建逆问题。我们已知的是物体在180个离散角度下的投影数据(即正弦图),目标是反演出物体内部的二维吸收率分布函数 $f(x,y)$。此过程的数学实质是求解拉东逆变换。直接应用拉东逆变换的理论——傅里叶中心切片定理,在离散数据和有限角度的实际情况下会遇到挑战。该定理揭示了投影数据的一维傅里叶变换对应于物体二维傅里叶谱的中心切片,但这种极坐标下的采样方式导致频域数据分布不均,直接进行傅里叶逆变换会因高频信息缺失而产生严重的模糊和星状伪影。因此,选择一种既能有效利用投影数据,又能克服上述理论缺陷的鲁棒重建算法,是解决问题的关键所在。

\subsection{问题四}

问题四旨在对前述研究进行深化与拓展,包含两个核心层面:一是评估与分析问题一中所建立的参数标定模型的精度与稳定性;二是在此分析的基础上,提出并论证一种性能更优的新标定模板设计方案。这一问不仅要求我们对模型误差的来源、传播机理及其对重建图像质量的影响有深刻的理解,还考验我们根据误差分析结果进行逆向工程设计的能力。其核心挑战在于,如何将旋转中心、探测器间距和扫描角度等参数的估计误差与最终图像的伪影(如环状伪影、几何畸变)建立定性乃至定量的联系,并以此为准则,设计出能够在投影数据中提供更强、更明确、解耦性更好的几何约束的新模板,从而从根本上提升标定过程的精度与鲁棒性。