\section{问题分析}
从实际问题到模型建立是一种从具体到抽象的思维过程,问题分析这一部分就是沟通这一过程的桥梁,因为它反映了建模者对于问题的认识程度如何,也体现了解决问题的雏形,起着承上启下的作用,也很能反应出建模者的综合水平。
这部分的内容应包括:题目中包含的信息和条件,利用信息和条件对题目做整体分析,确定用什么方法建立模型,一般是每个问题单独分析一小节,分析过程要简明扼要, 不需要放结论。
建议在文字说明的同时用图形或图表(例如流程图)列出思维过程,这会使你的思维显得很清晰,让人觉得一目了然。
(注意:问题分析这一部分放置的位置比较灵活,可以放在问题重述后面作为单独的一节(见到的频率最高),也可以放在模型假设和符号说明后面作为单独的一节,还可以针对每个问题将其写在模型建立中。具体可以看视频讲解)
\subsection{问题一}


\subsection{问题二}


\subsection{问题三}

