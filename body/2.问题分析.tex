\section{问题分析}
\subsection{问题一}
问题一的核心任务是利用已知的标定模板(图2)及其投影数据(附件2),
对CT系统的三个关键几何参数——旋转中心坐标、探测器单元间距和180个扫描角度 —— 进行精确标定。
解决此问题的基本思路是,建立未知参数与投影数据可观测特征之间的数学关系,
通过对比理论模型与实际数据来反向求解参数。\par
本文计划采用分步求解的策略:
首先,利用模板中小圆其在任何角度下的投影宽度均为直径的几何特性,
通过在投影数据中识别并测量其所占据的探测器单元数量,即可标定出探测器单元的物理间距。
然后,我们需要应用投影质心轨迹法。
该方法指出,物体投影的质心位置会随着扫描角度的变化呈现标准的正弦曲线轨迹。
本文通过采用吸收率中心来
计算出附件2中每一行投影数据的质心位置,
然后利用最小二乘法将这180个质心点拟合成一条正弦曲线。
该正弦曲线的振幅、相位和直流偏置三个参数,与模板的质心、系统旋转中心以及初始扫描角度直接关联,
通过求解这一方程组,便可同时确定旋转中心的精确坐标和扫描的初始角度,进而推算出全部180个扫描方向。

\subsection{问题二}


\subsection{问题三}

