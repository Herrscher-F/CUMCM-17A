\section{问题分析}
\subsection{问题一}
针对此题的系统参数标定,核心是求解一个典型的反问题。
现已知标定模板的精确几何构造和实际CT扫描投影数据,本文的策略是建立一个能够模拟CT扫描过程的理论投影模型。
利用该模型,将以待标定的系统参数 —— 旋转中心坐标 $(x_c, y_c)$、探测器单元间距 $d$ 
和180个扫描角度 $\{\theta_k\}$作为输入,通过解析几何计算出X射线穿过模板的弦长,得到理论投影数据。
随后将这个理论数据与附件2中的实际测量数据进行比对,通过定义一个量化两者差异目标函数(如均方误差),
将参数标定问题转化为一个最优化问题。
最终,我们将采用成熟的优化算法,如Nelder-Mead单纯形法,
来搜索使目标函数最小化的最优参数组合,从而完成系统标定。

求解拉东逆变换。

\subsection{问题二}


\subsection{问题三}

