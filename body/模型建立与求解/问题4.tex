\subsection{问题四模型建立与求解}
\subsubsection{模型建立}
问题4要求对问题1中的参数标定过程进行灵敏度分析,主要包括精度和稳定性两个方面。在开始分析之前,我们首先需要明确参数标定过程中的误差来源、影响以及传播路径。
在问题1中,误差的主要来源是离散化采样,包括空间坐标采样($x_k,y_l$)、探测器间距采样($\rho_i$)和角度采样($\theta_j$)。
其中角度采样和探测器间距采样共同影响了探测器间距$\Delta\rho$的求解,空间坐标采样主要影响旋转中心$(x_c,y_c)$和旋转角度$\theta_j$的求解。
\par
进一步的,这些误差的传播又会产生如下影响:
\begin{itemize}
    \item \textbf{旋转中心误差:} 如果估计的旋转中心 $(x_c,y_c)$ 与真实值存在偏差,会导致投影数据的不一致性。特别是,相隔 $180°$ 的两个投影(例如 $\theta$ 和 $\theta+180°$)在理想情况下应该是关于旋转中心对称的镜像,但中心误差会破坏这种对称性。当这些不一致的投影被反投影时,它们无法完美地对齐和叠加,从而产生模糊和特征性的\textbf{环状或条纹伪影}。  
    \item \textbf{探测器间距误差:} $\Delta d$ 的误差主要导致\textbf{几何尺寸的缩放失真}。如果估计的 $\Delta d$ 偏大,重建出的物体会比实际尺寸大;反之则偏小。  
    \item \textbf{角度误差:} 初始角度 $\theta_0$ 或角度增量 $\Delta\theta$ 的误差会导致\textbf{几何扭曲和边缘模糊},因为反投影时会将数据添加到错误的方向上。
\end{itemize}

这一分析同样为新模板的设计提供了思路:
通过对误差来源的分析,再结合相关的参考文献,我们可以总结出改进标定精度的关键在于设计一个能够提供更强、更明确几何约束的标定模板,而一个优秀的标定模板应遵循以下设计原则:
\begin{itemize}
    \item \textbf{非对称性:} 原始模板具有高度的对称性,这可能导致在某些角度下的投影特征不唯一,给角度和旋转中心的精确拟合带来模糊性。一个非对称的模板在每个投影角度下都会产生独特的投影,从而为参数求解提供更强的约束 。  
    \item \textbf{尖锐的局部特征:} 原始模板由大块的几何图形构成,其边缘在含噪的投影中可能变得模糊,使得质心或边界的定位精度不高。使用尺寸小、吸收率高(高对比度)的标记点(如金属微珠)作为特征,可以在投影数据中产生尖锐的脉冲信号,其质心位置可以被非常精确地定位,极大地提高了对噪声的鲁棒性。  
    \item \textbf{优化的几何布局:} 标记点不应随意放置。将它们分布在不同的半径和方位上,可以为旋转中心的求解提供更好的几何杠杆作用。比如沿坐标轴放置标记点就能够有助于解耦$x$和$y$方向的误差。  
    \item \textbf{多密度材料:} 在模板中集成多种已知不同吸收率的材料,不仅可以用于几何标定,还可以同时标定系统的灰度响应曲线,甚至用于校正X射线束硬化等非线性效应。
\end{itemize}


\subsubsection{模型求解}
\textbf{灵敏度检验:}
为量化评估模型的稳定性和精度,可以采取以下步骤:
\begin{itemize}
    \item \textbf{精度验证:} 使用第一部分标定的参数,对已知的标定模板(附件1)自身进行FBP重建。将重建图像与理论上的模板图像进行比较,计算均方误差(MSE)或结构相似性指数(SSIM)等指标,量化重建的保真度。(哪个指标表现好就用哪个指标)  
    \item \textbf{稳定性测试:} 在标定出的参数上人为引入微小的随机扰动(例如,给 $x_c$ 增加0.1 mm的误差),然后用受扰动的参数重新进行重建。观察重建图像质量的退化程度,特别是伪影的出现情况。通过多次实验,可以评估模型对各类参数误差的敏感度,即稳定性。
\end{itemize}


\textbf{设计方案:}  
基于建模过程中的设计原则,我们提出一种新型标定模板。该模板由一个低吸收率的基座(如有机玻璃圆盘)和嵌入其中的四个高密度、小尺寸的钨钢微珠构成。这四个微珠的布局是非对称的,具体来说:
\begin{itemize}
    \item 三个微珠构成一个不等边直角三角形,顶点分别位于 $(x_1,y_1)$, $(x_2,y_2)$, $(x_3,y_3)$。  
    \item 第四个微珠放置在离中心较远的位置 $(x_4,y_4)$,以打破任何潜在的对称性,并为旋转半径的确定提供长力臂。
\end{itemize}
使用这种新模板,标定算法将变得更为强大和稳定:
\begin{itemize}
    \item \textbf{特征识别:} 在正弦图中,四个高对比度的微珠会形成四条清晰、独立的、高信噪比的正弦轨迹。  
    \item \textbf{独立拟合与联合优化:}  
    \begin{itemize}
        \item 对每一条轨迹 $k$($k=1,2,3,4$)单独进行正弦拟合,得到其振幅 $A_k$、相位 $\phi_k$ 和偏置 $C_{offset,k}$。  
        \item 建立一个联合优化模型。该模型的变量是全局的系统参数(旋转中心 $(x_c,y_c)$,初始角度 $\theta_0$ 等)。目标函数是最小化所有四条轨迹的理论模型与实际数据之间的总误差。  
        \item 该模型还包含强几何约束:任意两个微珠 $k_1,k_2$ 在重建空间中的距离必须等于其已知的物理距离 $D_{12}=\sqrt{(x_{k_1}-x_{k_2})^2+(y_{k_1}-y_{k_2})^2}$。这些约束关系可以通过拟合出的振幅和相位参数来表达。  
    \end{itemize}
\end{itemize}


