\subsection{问题二}

此问题的核心任务是进行图像重建,即利用问题一中精确标定的系统几何参数(旋转中心、探测器间距、扫描角度)
以及附件3提供的未知介质的投影数据(正弦图),反向求解出该介质在托盘区域内的二维吸收率分布函数 $f(x,y)$。
\par
从数学角度看,此过程本质上是求解一个逆问题,具体而言是执行逆Radon变换。
CT系统测得的投影数据,是在一系列角度下对函数 $f(x,y)$ 进行的线积分,
因此,图像重建就是要从其Radon变换(投影数据)中恢复原始函数 $f(x,y)$。

\subsubsection{傅里叶中心切片定理}
一个二维函数 $f(x,y)$ 在角度 $\theta$ 方向上的一维投影 $p(\rho,\theta)$,其一维傅里叶变换$P(\omega,\theta)$,
恰好等于该二维函数 $f(x,y)$ 的二维傅里叶变换 $F(u,v)$ 沿着穿过原点、与投影方向成相同角度 $\theta$ 的一条直线(即一个“切片”)上的值。
用算子表示为:
$$\mathcal{F}_1\{p(\rho,\theta)\}=F(\omega\cos\theta,\omega\sin\theta)$$
其中 $\mathcal{F}_1$ 表示一维傅里叶变换,$\omega$ 是频率变量。\par

该定理说明,通过采集各个角度(即$\{\theta_j|j\in [180] \}$)的投影,并对它们进行一维傅里叶变换,
我们就可以得到物体的二维傅里叶谱在极坐标下的一系列径向采样。
理论上只要采集了足够多的角度,就可以填满整个二维傅里叶空间,
再通过二维傅里叶逆变换,即可恢复出原始图像 $f(x,y)$。\par

然而,在实际过程中,如果直接进行反投影,最终重建出的图像会存在严重的伪影,或者说,吸收率数据中会存在大量的噪声。究其原因,是由于二维傅里叶谱中的各角度切片采样并不足以提供重建所需的全部信息,而且这一问题并不会由于采样角度的增加而得到大幅度的改善。我们知道,对于2D-DFT,其低频区域靠近原点,代表图像总体结构;而高频区域远离原点,代表图像细节。无论采样角度如何增加,随着直线的向外放射,其采样点会越来越稀疏,导致高频信息丢失。\par
简单来说,这种在二维傅里叶谱中的角度切片采样是一种非均匀密度采样,存在低频区过采样、高频区欠采样的问题。为了缓解这一问题带来的影响,我们采用滤波反投影算法(FBP)进行重建。\par
FBP算法的核心思想是,在反投影之前,对每个角度切片进行滤波,以增强高频信息,从而改善重建质量。
其中,滤波函数通常采用斜坡滤波器,其表达式为:
$$H(\omega)=|\omega|$$
但高频区域除了图像细节以外,还包括各种噪声。为了防止高频噪声被过度放大,我们在斜坡滤波器的基础上乘上一个窗函数,得到:
$$H_{win}(\omega)=|\omega|\cdot W(\omega)$$
本题采用汉明窗函数,其表达式为:
$$W(\omega) = 0.54 - 0.46 \cos(\frac{2\pi \omega}{N-1})$$
其中,$N$ 是采样点数。\par

\subsubsection{模型求解}
现在开始对模型进行求解,具体过程如下:
\begin{enumerate}
    \item 使用附件3的投影数据,并结合问题1计算出的参数,将其转换为规范化正弦图 $p(\rho_i,\theta_j)$。其中,
    \begin{itemize}
        \item 角度坐标 $\theta_j$:$\theta_j=\theta_0+(j-1)\Delta\theta$,其中 $j=1,...,180$,$\theta_0$ 是初始角度,$\Delta\theta$ 是角度增量,本题为$1°$。
        \item 空间坐标 $\rho_i$:$\rho_i=(i-i_c)\cdot\Delta d$,其中 $i=1,...,512$,$i_c$ 是旋转中心在探测器上的投影像素索引,可以通过 $C_{offset}$ 计算得到。
    \end{itemize}
    \item 对每一组投影数据 $p(\rho,\theta_j)$(正弦图的每一行)进行一维快速傅里叶变换(FFT),得到其频域表示 $P(\omega,\theta_j)$。将投影的傅里叶变换与加窗的斜坡滤波器相乘:
    $$P_{filtered}(\omega,\theta_j)=P(\omega,\theta_j)\cdot H_{win}(\omega)$$。
    \item 对滤波后的频域数据 $P_{filtered}(\omega,\theta_j)$ 进行一维傅里叶逆变换(IFFT),得到滤波后的投影数据 $p_{filtered}(\rho,\theta_j)$。
    \item 创建一个256×256的空白图像矩阵 $f(x,y)$,并初始化为0。该矩阵的坐标系与托盘坐标系对应。
    \item 对图像矩阵中的每一个像素 $(x,y)$ 进行遍历。
    \item 对于每一个像素 $(x,y)$,再对所有180个投影角度 $\theta_j$ 进行遍历。
    \item 计算像素 $(x,y)$ 在旋转了角度 $\theta_j$ 后的坐标系中,到旋转中心的投影距离 $\rho'$。考虑到旋转中心 $(x_c,y_c)$,坐标变换应为:
       $$\rho'=(x-x_c)\cos\theta_j+(y-y_c)\sin\theta_j$$
    \item 计算出的 $\rho'$ 通常不是探测器采样点的整数倍。因此,需要在滤波后的投影数据 $p_{filtered}(\rho,\theta_j)$ 中,使用线性插值来估计 $\rho'$ 位置的投影值。
    \item 将插值得到的投影值累加到像素 $(x,y)$ 上:
       $$f(x,y)=f(x,y)+\text{Interpolate}(p_{filtered}(\cdot,\theta_j),\rho')\cdot\frac{\pi}{180}$$
       最后的因子 $\frac{\pi}{180}$(即 $\Delta\theta$ 的弧度值)是离散求和对连续积分的近似修正。
    \item 当所有像素都累加了所有角度的贡献后,得到的矩阵 $f(x,y)$ 就是重建出的物体断层图像。
    \item 通过对重建出的$f(x,y)$进行可视化,可以直观地判断未知介质的几何信息。通过设定一个合适的阈值,可以将物体与背景分离,从而清晰地观察其轮廓。计算二值化图像的质心(我们假设该物体的质量分布是均匀的),可以确定其在托盘中的位置。
\end{enumerate}


