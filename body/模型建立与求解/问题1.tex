\subsection{问题一模型建立与求解}
\subsubsection{模型建立}
CT成像过程所利用的数学模型是拉东变换。拉东变换是一种积分变换,能够将一个二维函数投影到一系列的线积分上。简单来说,拉东变换就是把一个物体映射为在其不同角度下的“投影”的集合。
在本题中,样品内部的吸收率分布可以用一个二维函数 $f(x,y)$ 来表示。当一束平行X射线以角度 $\theta$ 穿过该物体时,在探测器上距离原点为 $\rho$ 的位置所接收到的信号强度$p(\rho,\theta)$,正是吸收率函数 $f(x,y)$ 沿着直线 $x\cos\theta+y\sin\theta=\rho$ 的线积分,即:
$$p(\rho,\theta)=\int_{-\infty}^{\infty}\int_{-\infty}^{\infty}f(x,y)\delta(x\cos\theta+y\sin\theta-\rho)dxdy$$
其中 $\delta(\cdot)$ 是狄拉克函数,它确保了只有当 $x\cos\theta+y\sin\theta=\rho$ 时,$f(x,y)$ 才对积分有贡献。所有投影值 $p(\rho,\theta)$ 构成的二维图像被称为正弦图。

在现实中,CT系统所用的探测器数量显然是有限的,且吸收率函数$f(x,y)$往往也需要进行离散化采样。因此,我们需要先将拉东变换的连续形式离散化,将积分式改为有限求和的形式。此外,狄拉克函数也应该进行离散化处理,变为单位脉冲序列$\delta[\cdot]$:
$$p(\rho_i,\theta_j)=\sum_{k=1}^{N}\sum_{l=1}^{M}f(x_k,y_l)\delta[x_k\cos\theta_j+y_l\sin\theta_j-\rho_i]$$

在本题中$M=N=256$,$i \in [1,512]$,$j \in [1,180]$,且有以下公式:

\begin{equation*}
    \begin{aligned}
        \rho_i &= i\Delta\rho, \quad i=1,2,\cdots,512 \\
        \theta_j &= j\times 1^\circ+\theta_0, \quad j=1,2,\cdots,180
    \end{aligned}
\end{equation*}

\subsubsection{模型求解}
\textbf{探测器间距$\Delta\rho$求解}
\par
第一问提供的模板中包含一个半长轴为40mm,半短轴为15mm的椭圆与一个半径为4mm的实心圆。二者中心相距45mm。
利用这些几何数据,再根据附件2提供的接收信息数据,我们可以计算出探测器间距$\Delta\rho$。
具体过程如下:
\begin{enumerate}
    \item 按照角度划分数据,共180组;
    \item 按照0与非0对每组的数据进行阈值划分;
    \item 利用梯度信息寻找物体边界;
    \item 根据边界信息计算两个模板物体的投影长度;
    \item 寻找特殊角度,使得投影长度最大或最小\footnote{椭圆的投影长度最大时,X射线方向垂直于椭圆长轴;椭圆的投影长度最小时,X射线方向垂直于椭圆短轴。由于旋转角度为180度,因此一定能够保证存在这两个特殊角度。};
    \item 利用几何信息计算探测器间距。
\end{enumerate}


\textbf{旋转中心$(x_0,y_0)$求解}
\par
一个刚性物体在围绕一个固定中心旋转时,其投影到一维探测器上的质量中心(质心)的运动轨迹是一条完美的正弦曲线。这条正弦曲线的振幅、相位和直流偏置量,唯一地决定了物体的质心相对于旋转中心的位置。
