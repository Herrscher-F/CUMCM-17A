\subsection{问题一模型建立与求解}
\subsubsection{模型建立}
模型建立是将原问题抽象成用数学语言的表达式,它一定是在先前的问题分析和模型假设的基础上得来的。因为比赛时间很紧,大多时候我们都是使用别人已经建立好的模型。这部分一定要将题目问的问题和模型紧密结合起来,切忌随意套用模型。我们还可以对已有模型的某一方面进行改进或者优化,或者建立不同的模型解决同一个问题,这样就是论文的创新和亮点。
\subsubsection{模型求解}
把实际问题归结为一定的数学模型后,就要利用数学模型求解所提出的实际问题了。一般需要借助计算机软件进行求解,例如常用的软件有Matlab, Spss, Lingo, Excel, Stata, Python等。求解完成后,得到的求解结果应该规范准确并且醒目,若求解结果过长,最好编入附录里。(注意:如果使用智能优化算法或者数值计算方法求解的话,需要简要阐明算法的计算步骤)
