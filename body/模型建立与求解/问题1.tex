\subsection{问题一}

\subsubsection{离散 Radon 变换模型}

CT成像的物理过程在数学上可以通过Radon变换进行描述。
Radon变换是一种将二维空间函数投影至其在线积分集合上的积分变换,
简单来说,就是一个物体映射在不同角度下的“投影”的集合。\par

在本题中,已知二维断层切片的吸收率分布是一个二维函数 $f(x,y)$ ,
当一束平行的X射线以角度 $\theta$ 穿过,位于探测器坐标系中 $\rho$ 位置的探测单元所接收到的信号强度$p(\rho,\theta)$,
即为函数 $f(x,y)$ 沿着直线 $x\cos\theta+y\sin\theta=\rho$ 的线积分,即:
$$p(\rho,\theta)=\int_{-\infty}^{\infty}\int_{-\infty}^{\infty}f(x,y)\delta(x\cos\theta+y\sin\theta-\rho)dxdy$$

其中 $\delta(\cdot)$ 是狄拉克函数,其性质确保了积分仅在指定的射线上进行。
仅当 $x\cos\theta+y\sin\theta=\rho$ 时,$f(x,y)$ 才对积分有贡献。
所有投影值 $p(\rho,\theta)$ 构成的二维图像被称为正弦图。\par

在实际的数字成像系统中,探测器单元数量有限,且待重建的吸收率函数 $f(x,y)$ 在离散的像素网格上表示。
因此,必须对连续的Radon变换模型进行离散化。

本文将积分运算近似为有限求和,并将狄拉克函数替换为其离散形式——单位脉冲序列$\delta[\cdot]$。
假设图像空间被离散为 $N \times M$ 的像素网格,则在第 $j$ 个角度 $\theta_j$ 下,第 $i$ 个探测器单元 $\rho_i$ 的投影值可以表示为:
$$p(\rho_i,\theta_j)=\sum_{k=1}^{N}\sum_{l=1}^{M}f(x_k,y_l)\delta[x_k\cos\theta_j+y_l\sin\theta_j-\rho_i]$$

在本问题中,图像网格大小为 $M=N=256$,探测器单元序号$i \in [1,512]$,扫描角度序号$j \in [1,180]$。

探测器单元的物理位置 $\rho_i$ 可定义为:

\begin{equation*}
\rho_i = i\Delta\rho, \quad i=1,2,\cdots,512 \\
\end{equation*}

取$\theta_0$ 为旋转前的初始角度,系统参数的离散化关系为:

\begin{equation*}
\theta_j = \Delta\theta_j + \theta_{j-1}, \quad j=1,2,\cdots,180
\end{equation*}

题目提到系统绕某固定的旋转中心逆时针旋转 180 次,我们假设每次旋转角度 $\Delta\theta_j = 1^\circ$,
上式更变为:
\begin{equation*}
    \theta_j = j \times 1^\circ + \theta_0, \quad j=1,2,\cdots,180
\end{equation*}

其中,$\Delta\rho$ 为待求的探测器单元间距,
$\Delta\theta$ 为角度增量,
求解的核心任务便是通过标定,
精确确定参数 $\Delta\rho$、旋转中心 $(x_c, y_c)$ 以及初始角度 $\theta_0$。


\subsubsection{探测器间距求解}

探测器单元间距$\Delta\rho$的精确标定是后续图像重建的基础。
本文利用标定模板中已知的精确几何尺寸与附件2中实测投影数据(正弦图)之间的对应关系来求解该参数。
核心思想在于,通过识别特定投影角度下的模板投影宽度,
建立物理尺寸(单位:mm)与探测器阵列测量宽度之间的比例关系。\par


由椭圆的几何特性和圆的对称性可知,椭圆的投影宽度随扫描角度变化,而圆的投影宽度恒定。
因此,我们可以利用圆的投影宽度恒定这一特性来求解探测器间距。\par

通过遍历所有180个角度的投影数据,我们可以找到对应的最大投影宽度和最小投影宽度,
具体过程如下:\par
$step1.$ 按照角度划分数据,共180组;\par
$step2.$ 按照0与非0对每组的数据进行阈值划分;\par
$step3.$ 利用梯度信息寻找物体边界;\par
$step4.$ 根据边界信息计算投影长度;\par
$step5.$ 寻找始终不变的投影长度。(圆的投影长度恒定)\par
$step6.$ 计算平均值,得到探测器间距。


\subsubsection{旋转中心求解}

一个刚性物体在围绕一个固定中心旋转时,其投影到一维探测器上的质量中心的运动轨迹是一条正弦曲线。这条正弦曲线的振幅、相位和直流偏置,唯一地决定了物体的质心相对于旋转中心的位置。利用质心轨迹来求解旋转中心的方法被称为“投影质心轨迹法”\par
在本题中,我们将投影的“质量中心”替换为“吸收率中心”,并利用吸收率中心的运动轨迹来求解旋转中心。\par
具体过程如下:
\begin{enumerate}
    \item \textbf{计算模板的吸收率中心:} 首先,在托盘坐标系(以托盘左下角为原点)中,计算标定模板的吸收率中心 $(x_m,y_m)$。
    \item \textbf{计算投影吸收率中心轨迹(TPC):} 对附件2中的180组投影数据,计算每一组投影的吸收率中心。对于第 $j$ 组投影(对应角度 $\theta_j$),其投影数据为 $p_j(i)$,$i=1,2,...,512$。其投影吸收率中心的像素位置 $C_j$ 计算如下:
       $$C_j=\frac{\sum_{i=1}^{512}i\cdot p_j(i)}{\sum_{i=1}^{512}p_j(i)}$$
       计算所有180个角度的投影吸收率中心,得到一个包含180个点的时间序列 $[C_1,C_2,...,C_{180}]$,这就是投影吸收率中心轨迹(TPC)。
    \item \textbf{正弦拟合:} 理论上,TPC应满足以下正弦模型:
       $$C(\theta)=A\sin(\theta+\phi)+C_{offset}$$
       使用非线性最小二乘法将计算得到的TPC数据点 $\{(\theta_j,C_j)\}_{j=1}^{180}$ 拟合到上述模型,求解出最佳的振幅 $A$、相位 $\phi$ 和直流偏置 $C_{offset}$。在拟合前,需要对角度 $\theta_j$ 做出初步假设,例如,假设为等间隔的 $0°,1°,...,179°$。这个初步假设的误差主要影响相位 $\phi$,后续可以被校正。
    \item \textbf{求解旋转中心 $(x_c,y_c)$:} 拟合得到的参数与旋转中心坐标 $(x_c,y_c)$ 之间存在直接的几何关系。设物体的吸收率中心为 $(x_m,y_m)$,旋转中心为 $(x_c,y_c)$。物体吸收率中心相对于旋转中心的坐标为 $(\Delta x,\Delta y)=(x_m-x_c,y_m-y_c)$。其极坐标表示为 $(R,\alpha)$,其中 $R=\sqrt{\Delta x^2+\Delta y^2}$,$\alpha=\text{atan2}(\Delta y,\Delta x)$。
       
       投影吸收率中心的物理位置 $C_{phys}(\theta)$ 与这些参数的关系是:
       $$C_{phys}(\theta)=R\sin(\theta-\alpha)+\rho_c(\theta)$$
       其中 $\rho_c(\theta)$ 是旋转中心在投影方向上的坐标。对于平行束,$\rho_c(\theta)$ 是一个常数,即旋转中心到探测器中心线的投影偏移。
       
       将像素位置 $C_j$ 转换为物理位置 $C_{phys,j}=(C_j-i_{center})\cdot\Delta d$,其中 $i_{center}$ 是探测器中心像素的索引(通常为256.5)。拟合后的模型变为:
       $$C_{phys}(\theta)=(A\cdot\Delta d)\sin(\theta+\phi)+(C_{offset}-i_{center})\cdot\Delta d$$
       通过对比,可以建立如下关系:
       \begin{itemize}
           \item 振幅:$A\cdot\Delta d=R=\sqrt{(x_m-x_c)^2+(y_m-y_c)^2}$
           \item 相位:$\phi=-\alpha=-\text{atan2}(y_m-y_c,x_m-x_c)$
           \item 偏置:$(C_{offset}-i_{center})\cdot\Delta d$ 代表了旋转中心在探测器坐标系中的平均偏移。
       \end{itemize}
\end{enumerate}

通过解这个方程组,并结合已知的模板吸收率中心 $(x_m,y_m)$,即可反解出旋转中心 $(x_c,y_c)$。








