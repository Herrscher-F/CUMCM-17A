\subsection{问题一}

\subsubsection{离散 Radon 变换模型}

CT成像的物理过程在数学上可以通过Radon变换进行描述。
Radon变换是一种将二维空间函数投影至其在线积分集合上的积分变换,
简单来说,就是一个物体映射在不同角度下的“投影”的集合。\par

在本题中,已知二维断层切片的吸收率分布是一个二维函数 $f(x,y)$ ,
当一束平行的X射线以角度 $\theta$ 穿过,位于探测器坐标系中 $\rho$ 位置的探测单元所接收到的信号强度$p(\rho,\theta)$,
即为函数 $f(x,y)$ 沿着直线 $x\cos\theta+y\sin\theta=\rho$ 的线积分,即:
$$p(\rho,\theta)=\int_{-\infty}^{\infty}\int_{-\infty}^{\infty}f(x,y)\delta(x\cos\theta+y\sin\theta-\rho)dxdy$$

其中 $\delta(\cdot)$ 是狄拉克函数,其性质确保了积分仅在指定的射线上进行。
仅当 $x\cos\theta+y\sin\theta=\rho$ 时,$f(x,y)$ 才对积分有贡献。
所有投影值 $p(\rho,\theta)$ 构成的二维图像被称为正弦图。\par

在实际的数字成像系统中,探测器单元数量有限,且待重建的吸收率函数 $f(x,y)$ 在离散的像素网格上表示。
因此,必须对连续的Radon变换模型进行离散化。

本文将积分运算近似为有限求和,并将狄拉克函数替换为其离散形式——单位脉冲序列$\delta[\cdot]$。
假设图像空间被离散为 $N \times M$ 的像素网格,则在第 $j$ 个角度 $\theta_j$ 下,第 $i$ 个探测器单元 $\rho_i$ 的投影值可以表示为:
$$p(\rho_i,\theta_j)=\sum_{k=1}^{N}\sum_{l=1}^{M}f(x_k,y_l)\delta[x_k\cos\theta_j+y_l\sin\theta_j-\rho_i]$$

在本问题中,图像网格大小为 $M=N=256$,探测器单元序号$i \in [1,512]$,扫描角度序号$j \in [1,180]$。

探测器单元的物理位置 $\rho_i$ 可定义为:

\begin{equation*}
\rho_i = i\Delta\rho, \quad i=1,2,\cdots,512 \\
\end{equation*}

取$\theta_0$ 为旋转前的初始角度,系统参数的离散化关系为:

\begin{equation*}
\theta_j = \Delta\theta_j + \theta_{j-1}, \quad j=1,2,\cdots,180
\end{equation*}

将此整体模拟成一个二维平面内的模板系统,由一个中心位于原点、长轴为 $a$、短轴为 $b$ 的椭圆和一个半径为 $r$、圆心位于 $(m, 0)$ 的圆构成。当一束与水平方向成夹角 $\theta$ 的平行光照射该模板时,两个图形将在垂直于光线方向的探测器平面上形成阴影。设光线方向的单位向量为 $\mathbf{d} = (\cos\theta, \sin\theta)$,则垂直于光线的探测器方向为单位向量 $\mathbf{n} = (-\sin\theta, \cos\theta)$。

椭圆在探测器方向上的投影为对称区间,其半宽度为:
\[
w_e(\theta) = \sqrt{b^2 \sin^2\theta + a^2 \cos^2\theta}
\]

圆心在投影方向上的坐标为:
\[
p_c = -m \sin\theta
\]

其对应的投影区间为 $[p_c - r,\ p_c + r] = [-m \sin\theta - r,\ -m \sin\theta + r]$。将椭圆和圆在探测器方向上的投影视为两个区间,则它们的总阴影长度为这两个区间的并集长度。可以统一表示为如下解析表达式:
\begin{equation}
\begin{split}
L_{\text{total}}(\theta) =\ & \max\left( \sqrt{b^2 \sin^2\theta + a^2 \cos^2\theta},\ -m \sin\theta + r \right) \\
& - \min\left( -\sqrt{b^2 \sin^2\theta + a^2 \cos^2\theta},\ -m \sin\theta - r \right)
\end{split}
\end{equation}











