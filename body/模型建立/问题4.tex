\subsection{问题四}
\subsubsection{精度与稳定性分析}
对问题一中标定方案的精度与稳定性进行分析,其前提是深入理解标定过程中的误差来源及其传播路径。
在我们的模型中,误差主要源于对连续物理世界的离散化采样,具体包括:
图像空间网格的离散化、探测器单元的离散化以及扫描角度的离散化。
这些固有的离散化误差会直接或间接地影响我们对系统参数 
—— 旋转中心包括空间坐标采样($x_k,y_l$)、探测器间距采样($\rho_i$)和扫描角度($\theta_j$)的估计精度。
其中角度采样和探测器间距采样共同影响了探测器间距$\Delta\rho$的求解,
空间坐标采样主要影响旋转中心$(x_c,y_c)$和旋转角度$\theta_j$的求解。
\par
这些参数的估计误差一旦产生,便会在后续的图像重建中被放大,并以特定的图像伪影形式表现出来。我们可以将误差的传播与影响归纳如下:
\begin{enumerate}[label=(\arabic*), left=0.5em]
    \item \textbf{旋转中心误差} \par
    如果估计的旋转中心 $(x_c,y_c)$ 与真实值存在偏差,会导致投影数据的不一致性。
    特别是,相隔 $180°$ 的两个投影(例如 $\theta$ 和 $\theta+180°$)在理想情况下应该是关于旋转中心对称的镜像,
    但中心误差会破坏这种对称性。当这些不一致的投影被反投影时,它们无法完美地对齐和叠加,
    从而产生模糊和特征性的环状或条纹伪影。  
    \item \textbf{探测器间距误差} \par
    $\Delta d$ 的误差主要导致几何尺寸的缩放失真。
    如果估计的 $\Delta d$ 偏大,重建出的物体会比实际尺寸大;反之则偏小。  
    \item \textbf{角度误差} \par
    初始角度 $\theta_0$ 或角度增量 $\Delta\theta$ 的误差会导致几何扭曲和边缘模糊,因为反投影时会将数据添加到错误的方向上。
\end{enumerate}


\subsubsection{新标定模板的设计原则}
上述误差分析不仅揭示了原标定方案的潜在弱点,也为设计性能更优的新模板提供了明确的指导方向。
一个优秀的标定模板,其设计的核心目标是能够在其投影数据中为待标定参数提供更强、更明确且耦合度更低的几何约束。
结合相关文献与我们的分析,我们总结出以下关键设计原则:
\begin{enumerate}[label=(\arabic*), left=0.5em]
    \item \textbf{非对称性} \par 
    原始模板具有高度的对称性,这可能导致在某些角度下的投影特征不唯一,给角度和旋转中心的精确拟合带来模糊性。一个非对称的模板在每个投影角度下都会产生独特的投影,从而为参数求解提供更强的约束 。  
    \item \textbf{尖锐的局部特征} \par
    原始模板由大块的几何图形构成,其边缘在含噪的投影中可能变得模糊,使得质心或边界的定位精度不高。使用尺寸小、吸收率高(高对比度)的标记点(如金属微珠)作为特征,可以在投影数据中产生尖锐的脉冲信号,其质心位置可以被非常精确地定位,极大地提高了对噪声的鲁棒性。  
    \item \textbf{优化的几何布局} \par
    标记点不应随意放置。将它们分布在不同的半径和方位上,可以为旋转中心的求解提供更好的几何杠杆作用。比如沿坐标轴放置标记点就能够有助于解耦$x$和$y$方向的误差。  
    \item \textbf{多密度材料} \par
    在模板中集成多种已知不同吸收率的材料,不仅可以用于几何标定,还可以同时标定系统的灰度响应曲线,甚至用于校正X射线束硬化等非线性效应。
\end{enumerate}





