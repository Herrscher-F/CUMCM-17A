\subsection{问题四}
\subsubsection{模型建立}
问题4要求对问题1中的参数标定过程进行灵敏度分析,主要包括精度和稳定性两个方面。在开始分析之前,我们首先需要明确参数标定过程中的误差来源、影响以及传播路径。
在问题1中,误差的主要来源是离散化采样,包括空间坐标采样($x_k,y_l$)、探测器间距采样($\rho_i$)和角度采样($\theta_j$)。
其中角度采样和探测器间距采样共同影响了探测器间距$\Delta\rho$的求解,空间坐标采样主要影响旋转中心$(x_c,y_c)$和旋转角度$\theta_j$的求解。
\par
进一步的,这些误差的传播又会产生如下影响:
\begin{itemize}
    \item \textbf{旋转中心误差:} 如果估计的旋转中心 $(x_c,y_c)$ 与真实值存在偏差,会导致投影数据的不一致性。特别是,相隔 $180°$ 的两个投影(例如 $\theta$ 和 $\theta+180°$)在理想情况下应该是关于旋转中心对称的镜像,但中心误差会破坏这种对称性。当这些不一致的投影被反投影时,它们无法完美地对齐和叠加,从而产生模糊和特征性的\textbf{环状或条纹伪影}。  
    \item \textbf{探测器间距误差:} $\Delta d$ 的误差主要导致\textbf{几何尺寸的缩放失真}。如果估计的 $\Delta d$ 偏大,重建出的物体会比实际尺寸大;反之则偏小。  
    \item \textbf{角度误差:} 初始角度 $\theta_0$ 或角度增量 $\Delta\theta$ 的误差会导致\textbf{几何扭曲和边缘模糊},因为反投影时会将数据添加到错误的方向上。
\end{itemize}

这一分析同样为新模板的设计提供了思路:
通过对误差来源的分析,再结合相关的参考文献,我们可以总结出改进标定精度的关键在于设计一个能够提供更强、更明确几何约束的标定模板,而一个优秀的标定模板应遵循以下设计原则:
\begin{itemize}
    \item \textbf{非对称性:} 原始模板具有高度的对称性,这可能导致在某些角度下的投影特征不唯一,给角度和旋转中心的精确拟合带来模糊性。一个非对称的模板在每个投影角度下都会产生独特的投影,从而为参数求解提供更强的约束 。  
    \item \textbf{尖锐的局部特征:} 原始模板由大块的几何图形构成,其边缘在含噪的投影中可能变得模糊,使得质心或边界的定位精度不高。使用尺寸小、吸收率高(高对比度)的标记点(如金属微珠)作为特征,可以在投影数据中产生尖锐的脉冲信号,其质心位置可以被非常精确地定位,极大地提高了对噪声的鲁棒性。  
    \item \textbf{优化的几何布局:} 标记点不应随意放置。将它们分布在不同的半径和方位上,可以为旋转中心的求解提供更好的几何杠杆作用。比如沿坐标轴放置标记点就能够有助于解耦$x$和$y$方向的误差。  
    \item \textbf{多密度材料:} 在模板中集成多种已知不同吸收率的材料,不仅可以用于几何标定,还可以同时标定系统的灰度响应曲线,甚至用于校正X射线束硬化等非线性效应。
\end{itemize}





