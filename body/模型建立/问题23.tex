\subsection{问题二和问题三}

此问题的核心任务是进行图像重建,即利用问题一中精确标定的系统几何参数(旋转中心、探测器间距、扫描角度)
以及附件3提供的未知介质的投影数据(正弦图),反向求解出该介质在托盘区域内的二维吸收率分布函数 $f(x,y)$。
\par
从数学角度看,此过程本质上是求解一个逆问题,具体而言是执行逆Radon变换。
CT系统测得的投影数据,是在一系列角度下对函数 $f(x,y)$ 进行的线积分,
因此,图像重建就是要从其Radon变换(投影数据)中恢复原始函数 $f(x,y)$。

\subsubsection{傅里叶中心切片定理}
CT图像重建的理论基石是傅里叶中心切片定理(Fourier Slice Theorem),
它建立了物体空间域与其投影频域之间的关键桥梁。
\par

该定理指出, $f(x,y)$ 在角度 $\theta$ 方向上的一维投影 $p(\rho,\theta)$,
其一维傅里叶变换$P(\omega,\theta)$,
在数值上恰好等于该二维函数 $f(x,y)$ 的二维傅里叶变换 $F(u,v)$ 
在频域空间中、以相同角度$\theta$穿过原点的一个“切片”上的值。
用算子形式表达为:

\begin{equation*}
    \mathcal{F}_1\{p(\rho,\theta)\}=F(\omega\cos\theta,\omega\sin\theta)
\end{equation*}

其中 $\mathcal{F}_1$  代表一维傅里叶变换算子, $\omega$ 是一维空间频率。\par

该定理揭示了一条看似直接的重建路径:
通过采集物体在各个角度 $\{\theta_j|j\in [180] \}$ 的投影,
并对每一组投影数据进行一维傅里叶变换,即可获得物体二维傅里叶谱 $F(u,v)$ 在极坐标下的一系列径向采样。
理论上,只要采样角度足够密集,便能填充整个傅里叶空间,
再通过一次二维傅里叶逆变换,即可完美恢复出原始图像 $f(x,y)$。\par

\subsubsection{滤波反投影算法(FBP)}
然而,这条路径在实践中存在一个根本性的缺陷。
我们获得的傅里叶谱数据是在极坐标系下沿径向采样的,
而标准的二维傅里叶逆变换则需要在笛卡尔网格上进行。
这种坐标系的不匹配导致了采样点分布的非均匀性:
在靠近频域原点的低频区域,多个角度的采样点密集重叠,
造成了信息冗余(过采样);而在远离原点的高频区域,径向采样线之间的距离随频率增大而发散,
造成了信息缺失(欠采样)。若直接基于此非均匀数据进行反变换(其在空间域等价于简单的反投影),
高频信息的不足会导致图像边缘和细节的严重模糊,产生系统性的星状伪影。

为修正这一系统误差,滤波反投影(Filtered Back-Projection, FBP)算法应运而生。其核心思想是在执行反投影操作之前,先对每个角度的投影数据在频域上进行一次“滤波”操作,以补偿高频信息的不足,校正反投影带来的模糊效应。
理想的校正滤波器是斜坡滤波器(Ramp Filter),其频域表达式为:
\begin{equation*}
    H(\omega)=|\omega|
\end{equation*}
该滤波器的形态直观地反映了其功能:它线性地放大了高频分量 ($|w|$ 较大处),
同时相对抑制低频分量,从而在反投影前有效地增强了图像的细节信息。
\par

然而,斜坡滤波器在放大高频细节的同时,也会无差别地放大测量数据中固有的高频噪声。
为了在增强细节与抑制噪声之间取得平衡,工程实践中通常会将斜坡滤波器与一个平滑的窗函数 $W(w)$ 相乘。
本项目中,我们选用经典的汉明窗(Hamming Window),
它能够在有效保留边缘信息的同时,平滑地将最高频的噪声成分衰减至零。其表达式为:
\begin{equation*}
    W(\omega) = 0.54 - 0.46 \cos(\frac{2\pi \omega}{N-1})
\end{equation*}

其中,$N$ 是投影数据的采样点数。\par
因此,最终在FBP算法中使用的组合滤波器为 
\begin{equation*}
H_{win}(\omega)=|\omega|\cdot W(\omega)
\end{equation*}
经过此滤波函数处理后的投影数据再进行反投影,即可获得清晰、准确的重建图像。




